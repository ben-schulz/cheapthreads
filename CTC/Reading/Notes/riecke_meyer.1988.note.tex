\documentclass[a4paper,10pt]{article}

\usepackage{amsmath}
\usepackage{amssymb}
\usepackage{proof}
\usepackage{semantic}

%opening
\title{Reading Notes: "Continuations May Be Unreasonable", Meyer and Riecke (1988)}
\author{Benjamin Schulz}

\begin{document}

\maketitle

\begin{abstract}

\end{abstract}

\section{Overview}

\begin{itemize}
 \item{Applying the continuation-passing (CPS) transform to a term in the untyped lambda calculus may break observational congruence;}
\item{Introduction of a call/cc operator may also break congruence;}
\item{This complicates reasoning (but does not render it completely impossible: see ``Reasoning with Continuations: Full Abstraction for Models of Control'' by Sitaram and Felleisen [1990])}
\end{itemize}

\section{Initial Definitions and Notations}

The paper deals with a simply typed call-by-value (CBV) $\lambda$-calculus with arithmetic and recursion.  Let $Eval_v(M)$ denote the term produced by evaluating $M$ according to the appropriate operational semantics.  Let $\lambda_v \vdash M = N$ denote \emph{lambda convertability}, when $M$ reduces to $N$ under the usual procedures of $\lambda_v$.  Let $\overline{M}$ denote the CPS transform of $M$.\\
Informally, CPS preserves $\lambda_v \vdash M = N$, as well as the observable ``final output'' of evaluation.  (This is stated more precisely in a later section.)\\
\emph{The thesis of this paper is not that the CPS transform is incorrect.}\\
\\
\textbf{Definition:} Two terms $M$ and $N$ are \emph{observationally congruent}, denoted by $M \equiv_v N$ if they have the same behavior in all complete contexts under $\lambda_v$ convertability.\\
\\
To recapitulate: $\lambda_v$ convertibility holds only when $M$ reduces to $N$ through some number of reduction steps; observational equivalence holds only when $M$ and $N$ are interchangeable in all complete program contexts.\\
\\
\subsection{Example}
Consider the following terms:\\
\begin{equation}
 M_1 = \lambda a . \lambda b . \lambda c . (\lambda x . (b\ a)\ x)\ (c\ a)\\
\end{equation}
and
\begin{equation}
 M_2 = \lambda a . \lambda b . \lambda c .(b\ a)\ (c\ a)\\
\end{equation}

These two terms are not $\lambda_v$ convertible, although they are observationally congruent in all contexts under CBV.  $\lambda_v$ convertibility fails because of the CBV evaluation strategy, which will prevent the $\beta$-reduction of the body of $M_1$ which would otherwise result in a term identical to $M_2$.

These two terms are also inequivalent under CPS.  Consider the following evaluation context:

\begin{equation}
   \mathcal{C}[\bullet]= [ \bullet ] 1 (\lambda x . \Omega) (\lambda y . 
\mathcal{C}(\lambda x . 1))
\end{equation}

Recall that $\Omega \equiv (\textbf{Y} (\lambda f . \lambda x . f\ x))$, where $\textbf{Y} \equiv
 \lambda f . (\lambda x . f\ (x\ x)) (\lambda x . f\ (x\ x))$; $\mathcal{C}$ denotes the
call-with-current-continuation operator, with the standard reductions:

\begin{eqnarray}
(\mathcal{C} M)N \rightarrow \mathcal{C}\lambda k . M(\lambda f . \lambda k . (f N))\\
%
M(\mathcal{C} N) \rightarrow \mathcal{C} \lambda k . N (\lambda v . k (M v))\\
%
\mathcal{C}M \rhd M (\lambda x. \mathcal{A} x)\\
\end{eqnarray}

The CBV evaluation of $M_1$ gives:\\

\begin{eqnarray}
 (\lambda a . \lambda b . \lambda c . (\lambda x . (b\ a)\ x)\ (c\ a))\ 
1\ (\lambda x . \Omega)\ (\lambda y . \mathcal{C}(\lambda x . 1)) \rightarrow\\
% 
(\lambda b . \lambda c . (\lambda x . (b\ 1)\ x)\ (c\ 1))\ 
(\lambda x . \Omega)\ (\lambda y . \mathcal{C}(\lambda x . 1)) \rightarrow\\
%
(\lambda c . (\lambda x . ((\lambda x . \Omega)\ 1)\ x)\ (c\ 1))\ 
(\lambda y . \mathcal{C}(\lambda x . 1)) \rightarrow \\
%
((\lambda x . ((\lambda x . \Omega)\ 1)\ x)\ ((\lambda y . \mathcal{C} (\lambda x . 1))\ 1))\rightarrow\\
%
(\lambda x. ((\lambda x . \Omega)\ 1)\ x)\ ((\lambda y . \mathcal{C} (\lambda x . 1))\ 1)\rightarrow\\
%
(\lambda x. ((\lambda x . \Omega)\ 1)\ x)\ (\mathcal{C} (\lambda x . 1)) \rightarrow\\
%
\mathcal{C} \lambda k . (\lambda x . 1) (\lambda v . k\ (\lambda x. ((\lambda x . \Omega)\ 1)\ x)\ v))
%
\end{eqnarray}
\\
A CBV evaluation of $M_2$ gives:\\

\begin{eqnarray}
%
(\lambda a . \lambda b . \lambda c . (b\ a)\ (c\ a))\ 1\ (\lambda x . \Omega)\ (\lambda y .
\mathcal{C} (\lambda x . 1)) \rightarrow\\
%
(\lambda b . \lambda c . (b\ 1)\ (c\ 1))\ (\lambda x . \Omega)\ (\lambda y .
\mathcal{C} (\lambda x . 1)) \rightarrow \\
%
(\lambda c . ((\lambda x . \Omega)\ 1)\ (c\ 1))\ (\lambda y .
\mathcal{C} (\lambda x . 1)) \rightarrow \\
%
((\lambda x . \Omega)\ 1)\ ((\lambda y .
\mathcal{C} (\lambda x . 1))\ 1) \rightarrow \\
%
((\lambda x . \Omega)\ 1)\ 
(\mathcal{C} (\lambda x . 1)) \rightarrow \\
%
\Omega\ 
(\mathcal{C} (\lambda x . 1)) \rightarrow \\
%
\mathcal{C} \lambda k . (\lambda x . 1)\ (\lambda v . k\ (\Omega\ v))
\end{eqnarray}
\\
Choosing an initial continuation to instantiate $k$ easily shows from here that $M_2$ will diverge,
since $\Omega$ will eventually be evaluated as an argument, whereas $M_1$ will not diverge, since
$\Omega$ is contained in the body of an abstraction that will not be evaluated.\\

It is important to note that the context that distinguishes these two terms involves the $\mathcal{C}$
operator, suggesting that it is the addition of continuations to the language that induces
the inequivalence.\\
\\
\textbf{Exercise:} Show that $\equiv_c\ \subset\ \equiv_v$.\\
\\
Note that there are other $M$ and $N$ for which $\lambda_v \vdash M = N$ and $\lambda_c \vdash M = N$.
Consider the following:\\

\begin{eqnarray}
 P_1 = \lambda a . \lambda b . (\lambda x . x) ((\lambda y . y) (a\ b))\\
%
P_2 = \lambda a . \lambda b . (\lambda x . x) (a\ b)
\end{eqnarray}

The CPS transforms are:\\
\\
$
 \overline{P_1} =\\
\lambda k . k\ (\lambda a . \lambda k\prime .
k\prime (\lambda b . \overline{(\lambda x . x) ((\lambda y . y) (a\ b))})) =\\
%
$
\\
$\lambda k . k\ (\lambda a . \lambda k\prime .
k\prime (\lambda b .
\lambda k\prime\prime . \overline{(\lambda x . x)} (\lambda m .
\overline{((\lambda y . y) (a\ b))} (\lambda n . m n k\prime\prime))) =\\
$
%
\\
$\lambda k . k\ (\lambda a . \lambda k\prime .
k\prime (\lambda b .
\lambda k\prime\prime . (\lambda k\prime\prime\prime . 
\lambda k\prime\prime\prime (\lambda x . (\lambda k\prime\prime\prime\prime . k\prime\prime\prime\prime x)))\\
\indent (\lambda m .
(\lambda l . l (\overline{(\lambda y . y)} (\lambda m\prime . \overline{(a\ b)}
(\lambda n\prime . m\prime n\prime l))))
(\lambda n . m n k\prime\prime))) =\\
$
%
\\
$
\lambda k . k\ (\lambda a . \lambda k\prime .
k\prime (\lambda b .
\lambda k\prime\prime . (\lambda k\prime\prime\prime . 
\lambda k\prime\prime\prime (\lambda x . (\lambda k\prime\prime\prime\prime . k\prime\prime\prime\prime x)))\\
\indent (\lambda m .
(\lambda l . l (\lambda l\prime . l\prime (\lambda y . (\lambda l\prime . l\prime y))
(\lambda m\prime . \lambda l\prime\prime . (\lambda k . k a)\\
\indent \indent (\lambda m\prime\prime . 
(\lambda k . k b) (\lambda n\prime\prime . m\prime\prime n\prime\prime l\prime\prime)
(\lambda n\prime . m\prime n\prime l))))
(\lambda n . m n k\prime\prime))) =\\
$
\\
and:\\
\\
$
\overline{P_2} =
$
\\
$
\lambda k . k (\lambda a . (\lambda k\prime . k\prime (\lambda b . \overline{(\lambda x . x) (a\ b)})))
=
$
\\
$
\lambda k . k (\lambda a . (\lambda k\prime . k\prime (\lambda b .
(\lambda k\prime\prime . \overline{(\lambda x . x)} (\lambda m . \overline{(a\ b)}
(\lambda n . m n k\prime\prime))))
=
$
\\
$
\lambda k . k (\lambda a . (\lambda k\prime . k\prime (\lambda b .
(\lambda k\prime\prime . \\
\indent (\lambda l . l (\lambda x . (\lambda k . k x))) 
(\lambda m . (\lambda l\prime . (\lambda k . k a) (\lambda m\prime . (\lambda k . k b))
(\lambda n\prime . m\prime n\prime l\prime))
(\lambda n . m n k\prime\prime))))
$
\\
The following conjecture is offered:\\
\\
\textbf{Conjecture:} For some direct semantics $D |[ \bullet |] $, continuation semantics
$C |[ \bullet |]$, continuation transform $\overline{M}$ and observational congruence relation
$\equiv_c$ that uses a call/cc operator in contexts, then:\\

\[
\overline{M} \equiv_v \overline{N}\\
\iff D|[ \overline{M} |] = D|[ \overline{N} |]\\
\iff C|[ M |] = C |[ N |]\\
\iff M \equiv_C N
\]
\\
This conjecture is verified in ``Reasoning with Continuations II: Full Abstraction for Models of Control''
by Sitaram and Felleisen (1990).\\

The consequences of this result are important, as it demonstrates that observational congruence
in $\lambda_v$ of the image of terms under the continuation transform exactly coincides with
the observational equivalence relation in $\lambda_c$.  Moreover, the continuation semantics of $M$
corresponds to the denotation of the continuation transform of $M$.\\

\section{Details of the Continuation Transform}

The basic CPS transform on the core lambda calculus is:

\begin{tabular}[t]{lll}
 $\overline{x}$ &=& $\lambda k . k\ x$\\
$\overline{\lambda x . M}$ &=& $\lambda k . k\ (\lambda x. \overline{M})$\\
$\overline{M\ N}$ &=& $\lambda k . \overline{M}\ (\lambda m . \overline{N}\ (\lambda n . m\ n\ k))$\\
\end{tabular}
\\
\\
\textbf{Example:} The CPS transform of $M_0 = (\lambda x . x)\ 0$ is:\\
\begin{eqnarray*}
 \overline{M_0} =\\
\overline{(\lambda x . x)\ 0}=\\
\lambda k_0 . \overline{(\lambda x . x)}\ (\lambda m . \overline{0}\ (\lambda n . m\ n\ k_0)) =\\
\lambda k_0 . (\lambda k_1 . k_1\ (\lambda x . (\lambda k_2 . k_2 x)))\ 
(\lambda m . (\lambda k_3 . k_3\ 0)\ (\lambda n . m\ n\ k_0))\\
\end{eqnarray*}
\\
Passing in the initial continuation $id = \lambda z . z$ to instantiate $k_0$, the resulting term
evaluates as:\\

\begin{eqnarray*}
 (\lambda k_1 . k_1 (\lambda x . (\lambda k_2 . k_2 x)))\ 
(\lambda m . (\lambda k_3 . k_3 0)\ (\lambda n . m\ n\ id)) \rightarrow \\
%
(\lambda m . (\lambda k_3 . k_3 0)\ (\lambda n . m\ n\ id))\ 
(\lambda x . (\lambda k_2 . k_2 x)) \rightarrow \\
%
(\lambda k_3 . k_3\ 0) (\lambda n . (\lambda x. (\lambda k_2. k_2 x))\ n\ id) \rightarrow\\
%
(\lambda n . (\lambda x. (\lambda k_2. k_2 x))\ n\ id)\ 0 \rightarrow\\
%
((\lambda x. (\lambda k_2. k_2 x))\ 0\ id) \rightarrow\\
%
((\lambda k_2. k_2\ 0)\ id) \rightarrow\\
%
id\ 0 \rightarrow\\
%
0\\
\square
\end{eqnarray*}

It may be interesting to note that the reduction of the CPS-transformed term results in a sequence
of reductions each of the form: $(\lambda u. M)\ (\lambda v. N)$, so that $u$ determines the evaluation
position of $N$; in general, $N$ will be evaluated exactly when $v$ has been instantiated, and the
resulting term appears on the RHS of the next application.\\

It is also instructive to note that the CPS transform changes types; for a type $\alpha$,
and ground type $o$, define $\overline{\alpha}$ as:\\

\begin{tabular}[t]{lll}
 $\overline{o}$ &=& $o$\\
$\overline{(\alpha \rightarrow \beta)}$ &=& $\overline{\alpha}
  \rightarrow (\overline{\beta} \rightarrow o) \rightarrow o$\\
\end{tabular}
\\
\\
The effect on function types is transform a function accepting an argument of type $\alpha$
and returning a value of type $\beta$ into a function accepting a ``continuized'' $\alpha$ and
a function producing an intermediate result from $\beta$ and producing a ground object $o$.

\subsection{Properties of the Transform}

CBV convertability is preserved by CPS transform, as proven by Plotkin in ``Call-by-name, call-by-value
and the lambda calculus" (1975):\\
\\
\textbf{Theorem:} If $M$ and $N$ are closed and $\lambda_v \vdash M = N$, then $\lambda_v \vdash
\overline{M} = \overline{N}$.\\
\\
As such, application of the CPS transform does not disturb the reduction procedures of $\lambda_v$.
As such, optimizations correct for direct terms are also possible on CPS terms.
Note, however, that convertability does not coincide with observation congruence.\\
\\
\textbf{Theorem (Correctness):} If $M$ is a complete program and $c : o$, then:\\
\[
 Eval_v(M) = c \iff Eval_v (\overline{M} (\lambda x^o . x)) = c
\]
\\
Clearly, if $M$ evaluates to $c$, then $\overline{M}$ should also evaluate to $c$, otherwise the
transform would be of no use.

\section{Operational Semantics and CPS}

\textbf{Definition (Observational Congruence)}: $M : \alpha$ and $N : \alpha$ are
\emph{observationally congruent} if:\\
\[
 \lambda_v \vdash C[M] = c \iff \lambda_v \vdash C[N] = c
\]
for all contexts $C[\bullet]$ and ground constants $c$.$\square$\\
\\
In general, the exact meaning of observational congruence will depend upon the set of contexts over
which $C[\bullet]$ ranges.  In particular, two terms that are observationally congruent in $\lambda_v$
do not necessarily have CPS images that are also observationally congruent in $\lambda_v$.\\
\\
\textbf{Theorem (Congruence Not Preserved by CPS):} There exist two closed terms $M$ and $N$ such that
$M \equiv_v N$ and $\overline{M} \equiv_v \overline{N}$.\\
\\
The terms $M_1$ and $M_2$ above have this property.  They are clearly observationally congruent.
However, $\overline{M_1}$ and $\overline{M_2}$ are distinguished by the following context:\\
\begin{tabular}[t]{lll}
 $C[\bullet]$ &=& $[\bullet] N_0$\\
$N_0$ &=& $\lambda f . f\ 1\ N_1$\\
$N_1$ &=& $\lambda g . g\ (\lambda x . \lambda y. 1) N_2$\\
$N_2$ &=& $\lambda h . h\ (\lambda x . \lambda y. \Omega) (\lambda x . x)$\\
\end{tabular}
\\
In this context, $C[\overline{M_2}]$ terminates but $C[\overline{M_1}]$ does not.  As in the examples
above, termination or non-termination is determined by whether $\Omega$ is ever evaluated, which
in turn is determined by whether or not $\Omega$ ever appears alone in the argument position of a
matching evaluation pattern. $\square$\\

\end{document}
