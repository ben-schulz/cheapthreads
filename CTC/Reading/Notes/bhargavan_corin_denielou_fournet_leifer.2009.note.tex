\documentclass[a4paper,10pt]{article}
\usepackage[utf8x]{inputenc}

\usepackage{amsmath}
\usepackage{proof}

%opening
\title{Notes on "Cryptographic Protocol Synthesis and Verification for Multiparty Sessions",
by Bhargavan, Corin, Denielou, Fournet, and Leifer (2009)}
\author{Benjamin Schulz}

\begin{document}

\maketitle

\begin{abstract}

\end{abstract}

\section{Motivation: Compilation and Static Typing for Security}

Widely available protocols for encrypted communication operate a lower level than application programming.
As such, they are perfectly suitable for messages exchanged only between two URLs or IP addresses,
but problematic for applications programming, where information may be passed among multiple parties,
e.g. intermediate gateways, web servers.\\

The authors present a language for specifying multiparty sessions.  Control flow is given by asynchronous
message passing and includes a shared store with access controls.  This framework is applied to the
task for reasoning about security properties of the sessions.\\

An important object of this approach is to hide cryptographic security measures from the
application-level programmer, which gives the appearance of a single, uniform protocol to all
participants in the communication.\\

Informally, define \emph{compromised principals} as those whose keys are known to an adversary;
these may be malicious principals, or principals whose keys have been leaked.  All other principals
are called \emph{compliant}.  The security goal is to protect compliant principals from compromised
principals.  The authors verify automatically generated protocols by generalizing over security
properties of all runs.  (Note similarity to the trace methods we are already familiar with,
as in the various non-interference arguments.)\\

\subsection{Possible Future Work:}
Verify something similar to the session compiler presented in this paper using Coq.\\
\\
\subsection{Possible Future Work:}
Address questions such as liveness, resistance to traffic analysis, or DoS attacks.\\
\\
\subsection{Related Work:} See the paper for complete list, but note that virtually all of the related work
on this subject has been done in the last ten years, and most of that in the last five.


\section{Multiparty Sessions}

It is useful to consider sessions as directed graphs.  Each node corresponds to a principal,
and each edge corresponds to a message from its source to its destination node.  Nodes are labeled
with their role, either \textsf{c} for client or \textsf{w} for web service.  Edges are labeled
according to particular signals, e.g. \textsf{Request}, \textsf{Ack} or \textsf{Fault}.\\

Assume that each participant has its own local implementation and application code that handles the
sending and receiving of messages.\\

Sessions specify the pattern of a communication; execution consists of a walk of the session graph.
Execution terminates on reaching a node with no outgoing edges.\\

Graphs may contain a cycle, e.g. repeated request-retry messages.  Note that it is also important that
each session execution only has a single token; the set of permissible graphs thus excludes all graphs
containing a branch with a set of roles that is not a subset of its predecessors.  In essence, this
feature models a restriction on session information escaping from the parties expressly involved
by the specification.\\

Access control is imposed by static typing of a finite set of session variables.  In the graph
model, variables are written by the outgoing (source) edge and read by the incoming (destination) edge.
It should thus be straightforward to specify behaviors in a standard way, e.g. as a Hoare triple.
In particular, this specification is used to specify which variables may be examined or modified
by a given participant at a given stage of the communication.\\

\emph{Roles} are a special class of variables assigned to principals of the communication.
For instance, a particular message from A may have a final destination of C by way of B, so that
$A \mapsto \textsf{c}$, $B \mapsto \textsf{p}$, and $C \mapsto \textsf{w}$.  It is important to
assume that role variables are written only once, and must have been read by all principals involved
in the session before sending any message to a new participant.  Note that role assignments may
occur dynamically, i.e. may be filled out by as the execution progresses.  For instance, a proxy
may choose the final destination server, and thus be responsible for choosing a \textsf{w}.\\

By way of example, the authors indicate that their compiler should never allow a session wherein
a proxy directly forwards a request to any server of its choice.  The reasoning is that a compromised
proxy might pass the request to multiple servers and select among the replies.

\subsection{Formal Development}

A \emph{session graph} is a tuple:\\

\[
 G = (\mathcal{R}, \mathcal{V}, \mathcal{X}, \mathcal{L}, m_0 \in \mathcal{V}, \mathcal{E},
  R : \mathcal{V} \rightarrow \mathcal {R})
\]

where $\mathcal{R}$ is a finite set of roles; $\mathcal{V}$ is a finite set of vertices (nodes);
$\mathcal{X} = \mathcal{X}_{\textsf{d}} \uplus \mathcal{R}$, the disjoint union of data variables
$\mathcal{X}_\textsf{d}$ and roles $\mathcal{R}$; a set of distinct labels $\mathcal{L}$; an initial state node $m_0$,
and a set of labeled edges $(m, \tilde{x}, f, \tilde{y}, m\prime) \in \mathcal{E} \subseteq$
$ \mathcal{V} \times \mathcal{X}^* \times \mathcal{L} \times \mathcal{X}^* \times \mathcal{V}$, where
$\tilde{X}$ denotes a vector $(x_0 \cdots x_k)$, and where each variable occurs no more than once
in each of $\tilde{x}$ and $\tilde{y}$ (necessary for state to be well-defined); and $R$ maps nodes
to their roles in the current communication.\\

Edges are uniquely identified by their labels.  For an edge $(m, \tilde{x}, f, \tilde{y}, m\prime)$,
read and written variables are specifed by their position in the tuple, i.e. such that
$\textsf{write}(f) = \tilde{x}$ and $\textsf{read}(f) = \tilde{y}$.  Similar, the \emph{source}
and \emph{target} roles of the edge with label $f$ are defined as $\textsf{src}(f) = R(m)$ and
$\textsf{tgt}(f) = R(m\prime)$.\\

Let $\hat{f}$ denote an \emph{extended path}, which is path through the graph annotated with vectors
of variables, as $(\tilde{x_0}) f_0 \cdots (\tilde{x_k}) f_k$.\\

\section{The Session Programming Model}

In outline, the session specification is written as a description in a specification language, which
is compiled to generate a library module that implements cryptographic communications.  The resulting
code is verified by a static type checker.  Application code for each role is then written for each
distinct role.  Trusted, i.e. compliant principals use code thus produced; untrusted i.e. compromised
principals may also participate, however, using their own application code.\\

Informally, sessions are specified in a syntax that reflects the formal definition, that is, that
specifies variables, roles, etc.  Roles specify an internal choice (which will appear non-deterministic
to other participants) of messages to send, and an external choice (which will appear non-deterministic
to the noe itself) of replies that may arrive.  The authors present a syntax that is local to nodes,
but note that this is not incompatible with a global specification.  Note that the resulting
session graph is subject to certain well-formedness constraints.\\

The protocol module generated by the presented tool exports functions for each role, with each
function implementing the corresponding role process of sends and receives.  This function takes
the identifying informaion of the running principal as its argument (e.g. IP address, crypto keys).\\

\emph{Here's the interesting part: the role function is written in a continuation-passing style
which allows the application to bind variables and choose among messages in each state.}\\

\subsection{Type Checking}

Application code that is well-typed (according to an ordinary ML typing -- which should includes
the necessary type definitions) when incorporating the generated protocol module(s) is guaranteed
to comply with session discipline.  This does not preclude attacks by compromised principals; thus,
sessions are encrypted to ensure global session integrity, even in the presence of attempts by
adversaries to alter the session data.\\

To ensure that cryptographic obligations are met, the session graph is encoded as a series of logical
pre- and post-conditions on Tx/Rx actions by the protocol functions.  The authors use ML, extended
with refinement types; the resulting program is checked by a custom typechecker that invokes an
automated theorem prover.

\subsection{Possible Project:} Do this in Coq.\\


\section{Formal Statement: Session Integrity}

The main security theorem quantifies over all messages sent and received by compliant principals
during session execution, in terms of events emitted by the implementation:\\
\\
\textbf{Informal Thorem (Session Integrity):} All send (Tx) and receive (Rx) events emitted by
the implementation will correspond to correct session traces in all possible environments of
compromised principals.\\
\\
Let a verified $\tilde{S}$-system consist of the following modules:\\
\[
 Data, Net, Crypto, Prins, (S_i\_protocol)^{1 \le i \le k}, U
\]
\\
where $Data$, $Net$, and $Crypto$ are symbolic implementations of trusted libraries; $Data$ operates
on byte arrays and strings, and $Net$ performs networking operations.  $Crypto$ (obviously) handles
cryptogrpahic operations, and $Prins$ assigns cryptographic keys to principals.  $S_i\_protocol$ is
a verified module generated by the authors' compiler from a session description $S_i$; it is
typechecked against its refined type interfaces and those of the libraries.  $U$ denotes application
code, which may be the code run by an adversary.  For the purposes of this paper, assume it to be
arbitrary ML code, with access to any and all functions exported by the modules above, and to some
of the keys in $Prins$.\\

\subsection{Exercise:} Look up the definition of a \emph{Dolev-Yao adversary} and explain its
relevance to this model.\\
\\
\subsection{Moving Right Along ...}
The adversary U may instantiate any of the roles in the session, may intercept, modify, or send messages
on public channels, or perform cryptographic computations.  It is assumed, however, that U cannot
break the cryptography or guess secrets belonging to compliant principals.\\
A run on the $\tilde{S}$-system is a sequence of events logged during the execution of the system.
For each session, define three kinds of observable events:\\
\[
 Tx(f, a, \tilde{v}), Rx(f, a, \tilde{v}), Lx(a)
\]
respectively, send, receive and leak, where $f$ ranges over the session labels.  $Tx(f, a, \tilde{v})$
asserts that, in some possible run, principal $a$, acting in the source role of edge $f$ commits
to send a message labeled as $f$ with values $\tilde{v}$ as its written variables.
$Rx(f, a, \tilde{v})$ asserts that $a$, acting in the target role of $f$ and after examining the
cryptographic evidence, accepts a message $f$ with read variables $tilde{v}$.  $Lx(a)$ asserts
that $a$ has been compromised, which occurs when $U$ demands a key from $Prins$.  In a run where the
principal's keys are never accessed by $U$, $Lx$ should never occur, and the principal should be
considered compliant.  For a given run, a compliant event is a $Tx$ or $Rx$ whose first argument
has not been compromised by a $Rx$.\\

A \emph{session trace} is a sequence of $Tx$ and $Rx$ obtained by globally instantiating all bound
variables in a path through the session whose initial node is $m_0$.  Formally,\\
\\
\textbf{Definition (Session Trace):} The traces of a session $S$ are constructed as follows:
Denote $f_1 \cdots f_k$ denote an initial path of $S$ and let $\tilde{x}_i = \textsf{write}(f_i)$ and
$\tilde{y}_i = \textsf{read}(f_i)$ be the respective read and written variables for $f_i$.  Let
$(\alpha_i)$ denote a sequence of maps of variables in $\mathcal{X}$ to values, such that for each
$i$, $\alpha_i$ and $\alpha_{i+1}$ differ only on $\tilde{x}_i$.
The \emph{session trace} is then obtained by replacing each edge $f_i$ in the path with two events
$Tx(f_i, \alpha_i(\textsf{src}(f_i)), \alpha_i\tilde{x}_i)$ and
$Rx(f_i, \alpha_i(\textsf{tgt}(f_i)), \alpha_i\tilde{y}_i)$.\\

A compliant subtrace of $S \in \tilde{S}$ is a projection of $S$ in that discards non-compliant events.
As events are logged, session traces should capture all sequences of events in a partial run.  Values
in the session variables as recorded in each event should accord exactly with the variable rewrites
specified by the graph.\\

All of this provides for a formal statement of the security property,\\

\textbf{Theorem (Session Integrity):} Any run of a verified $\tilde{S}$-system includes a partition
of the compliant events that coincides with compliant subtraces of sessions in $\tilde{S}$.\\

In essence, compliant events of any run are interleavings of compliant events as they may occur along
execution paths.  Important to this assertion is the assumption that views of the session state must
be consistent among principals.  The force of this assertion comes from the fact that all principals
using the protocol implementations $S_i\_protocol$ generated and verified by the compiler will
have consistent views of the session state, and thus should be protected from $U$.\\

For example, consider the simple proxy session shown in the figure.  Suppose the client and proxy are
compliant, but the web service is not.  When the theorem holds, it must be that when the client
received a 'Reply' message from the web service, the proxy previously sent a 'Forward' or 'Resume'
request to the web service.  Importantly, the web service cannot reply to the client until its
negotiation with the proxy has concluded.\\

\section{Protocol Design and Cryptography}

Assume first that every pair of principals $a$ and $b$ shares keys for encryption, MAC using
symmetric cryptography.  Let $enc_b^a\{m\}$ and $mac_b^a\{m\}$ denote the encryption and MAC for
node $m$.  Initially, $a$ generates keys for $b$, encrypts, signs and sends them to $b$ using
asymmetric cryptography.  Each message consists of a header, a series of variable assignments,
and a series of MACs.  The header consists of a session label and session identifier, we well
as a message sequence number.  Variable assignments are contained in the message payload.  MACs
are used to authenticate the integrity of the global state.  Importantly, this includes roles
assigned to principals.\\

Session identifiers are cached and used to prevent replay attacks.  Each principal should be able to
determine, by inspecting the header of an incoming message, whether the message is an invitation to join
a session or is a continuation of a session in which the recipient is already a party.  Session data
is updated at each step, so that messages arriving out of order will be discarded.  Since session
data specifies the permissible structure of a complete communication, and since sessions are tracked
by each principal, attempts to mount an attack by sending unexpected messages will be rejected.\\

Each role is characterized by a particular control flow graph, wherein only certain messages will
be accepted by certain states.  The content of a given message is determined by the respective states
of the bound roles, and current knowledge of variable contents.  Each such state can then be indexed
by the role name of the sender and a path containing the last label sent by each of the roles,
and the last occurrence of each written variable.  These correspond to internal control states.\\

(See Corin et al. 2008 for more details on the determination and importance of control flow states
in the session graph.)\\

Control flow states, denoted by $\rho$, loosely correspond to the nodes of the original session graph.
Control flow analysis actually gives a refinement of the session graph, which distinguishes between
the first entry of a node, and subsequent entries occurring as iterations of a cycle.  The $\rho$ index
the Tx and Rx functions in the generated protocol module.  Formally,\\

\textbf{Definition (Internal Control Flow State):} An \emph{internal control flow state}, denoted by
$\rho$, is an extended path obtained by applying the following function to a path with initial node
$m_0$:\\
Let $\textsf{st}(\tilde{f}) = \textsf{filter}(\tilde{f}, \epsilon, \epsilon)$, where $\textsf{filter}$
is recursively defined as:\\
$\textsf{filter}(\epsilon, \tilde{z}, \tilde{r}) = \epsilon$\\
\\
\begin{math}
\textsf{filter}(\tilde{f}\ f, \tilde{z}, \tilde{r}) =
\begin{cases}
(\textsf{filter}(\tilde{f}, \tilde{x\prime} \tilde{z}, r \tilde{r})) (\tilde{x\prime})f
\ \ if\ r \notin \tilde{r}\\
(\textsf{filter}(\tilde{f}, \tilde{x\prime} \tilde{z}, \tilde{r})) (\tilde{x\prime})\ \ otherwise\\
\end{cases}
\end{math}
\\
where $r = \textsf{src(f)}$ and $\tilde{x\prime} = \textsf{write}(f) \not \tilde{z}$.\\
\\
Informally, \textsf{filter} makes a pass over an initial path, right to left, removing any labels sent
by roles encountered on the right (which are accumulated in $\tilde{r}$) and any written variables
(which are accumulated in $\tilde{z}$, keeping only the last occurrence of each variable and the last
label sent by each role.  The result is that an internal control flow state gives the position fo each
of the roles in the graph and, by interleaving variables and labels, the current knowledge of the
store as seen by each role.\\

The basic idea here is simply that the internal control flow state of each node can be determined
at each stage of the communication.  This is an analysis performed by the compiler and used to
generate appropriate sending and receiving functions.\\

Note, at this point, that the trusted libraries are not verified by this method and are independent
of the toolchain; they are assumed to be correct.

\emph{Does this present any problems for CPS, as source terms may vary under CPS transforms?}\\

\section{Sessions are Path Predicates}

The Session Integrity theorem may be proven by considering paths as predicates.  The outline of the
proof is as follows: Tx and Rx events are annotated with additional parameters, which are lifted to
session traces also.  For each session, a family of predicates is constructed that must be maintained
by a session implementation.  Typed interfaces are defined for the generated code, and it is shown
that if code meets the types, then the invariants expressed by the types are maintained.  Each
implementation is shown to be locally sequential, and the integrity of all code generated and
typechecked by the compiler is established.\\

Informally, the compiler generates predicates on pre- and post-conditions of global state
for each particular control flow state that must be satisfied in order to ensure the integrity of
the session protocol.  Whether this is the case is determined by the Z3 SMT solver.\\

\subsection{Possible Future Work:} This paper does not detail secrecy, as this involves application
code.  However, this would be a natural application of work on resumptions.\\






\end{document}
