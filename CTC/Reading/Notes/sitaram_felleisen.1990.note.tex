\documentclass[a4paper,10pt]{article}
\usepackage[utf8x]{inputenc}

\usepackage{amsmath}
\usepackage{amssymb}
\usepackage{proof}

%opening
\title{Notes on ``Reasoning with Continuations II: Full Abstraction for Models of Control'',
Sitaram and Felleisen (1990)}
\author{Benjamin Schulz}

\begin{document}

\maketitle

\begin{abstract}

\end{abstract}

\section{Introduction}

Certain functional languages, namely Scheme and Lisp, use control structures (i.e. continuations)
for the purpose of labeling control state at some arbitrary point, which allows more elaborate
program behavior, and facillitates passing results between widely separated control points.\\

Adding control facilities to these languages voids certain reasoning principles, namely observational
equivalence (as in Riecke 1988); the authors purport to show how to reason even in the presence of
such control facilities, and how to relate the resulting reasoning system to reasonining in the
setting of a direct calculus.\\

Operational equivalence as a reasoning tool is necessarily sensitive to language extensions;
as evidenced by the problems that result from adding the call/cc and abort operators to the usual
$\lambda_v$ calculus.  The need for a reasoning system that is fully abstract -- which allows
denotational reasoning, and thus compositionality, and thus induction -- motivates this work.\\

\section{$PCF_v-C$}

$PCF_v-C$ is an extension of call-by-value (CBV) PCF, with control facilities, simple types, ground
constants, conditionals, and fixpoints.  As is usually the case, $\lambda$-abstractions introduce
bound variables, with all unbound variables \emph{free}, and terms \emph{closed} whenever they
contain no free variables.\\

Evaluation proceeds from left to right by searching the argument term for a redex that matches
the RHS of one of the reduction rules (see Figure 1 in the text), which can be expressed as a 
grammar of \emph{evaluation contexts}:\\

\[
 E[] ::= []\ \vert\ V E[]\ \vert\ E[] M
\]

Note that this grammar succinctly expresses the left-to-right ordering of evaluation, as only values
($V$) may appear to the left of an $E[]$ and only terms $M$ may appear to the right.\\

Evaluation here directly reflects the reduction rules in its evaluation steps, as:\\

\[
 M \rightarrow M\prime \implies E[M] \rhd E[M']
\]

Let $eval(M) = \textsf{b} \iff M \rhd^* \textsf{b}$ denote the function that fully evaluates a term.\\
\\
\textbf{Definition (Operational Approximation):} For terms $M : t$ and $N : t$, $M$
\emph{operationally approximates} $N$, written $M \sqsubseteq_v N$ if for all contexts $C[]$,
$eval(C[M]) = \textsf{b} \implies eval(C[N]) = \textsf{b}$.  $M$ and $N$ are \emph{operationally
equivalent}, written $M \equiv_v N$ if $M \sqsubseteq_v N$ and $N \sqsubseteq_v M$.\\
\\
\textbf{Example:} Let $M_1 = \lambda f . \Omega$ and $M_2 = \lambda f . f \ulcorner 0 \urcorner \Omega$.
We wish to argue that $M_1 \equiv_v M_2$.  Let $C[]$ be any enclosing program context.  If $M_1$
appears in the function position, i.e. instantiates $V$ in the production $V E[]$ as a result of
evaluation, then clearly $M_2$ will also appear in the function position of the same context.
When the resulting application term is reduced, following evaluation of the argument, it will
clearly diverge since:\\
\[ M_1\ V = (\lambda f . \Omega)\ V = \Omega[f \mapsto V]\]\\

Which clearly diverges.  Similarly,\\

\[M_2\ V = (\lambda f . f\ \ulcorner 0 \urcorner\ \Omega)\ V = (V \ulcorner 0 \urcorner) \Omega \]\\

Which also diverges, since $\Omega$ now appears in the argument position and will thus be evaluated.
If $M_i$ never appears in the function position in $C[]$, then $\Omega$ will never be evaluated.
Moreover, since $M_i$ does not appear in the function position, any such $C[]$ that is closed and has
ground type must be unaffected by substituting $M_i$ for $M_j$, since the program must reduce to
a ground constant.  (Note that it is important that observational equivalence requires reduction
to the same ground constant; otherwise $C[] = (\lambda x . x) []$ would distinguish $M_1$ and $M_2$.)


\subsection{The Abort Operator}

The first of the extended control structures to be considered is the \emph{abort} operator,
$\mathcal{A}$.  The behavior of $\mathcal{A}$ is characterized by an additional stepping rule,\\
\[
 E[\mathcal{A} M] \rhd M
\]

Thus, $\mathcal{A}$ causes evaluation to immediately terminate with its argument as a result.
Let $\equiv_a$ denote observational equivalence in the language PCF+$\mathcal{A}$.\\
\\
\textbf{Example:} Let $M_1 = (\lambda f . \Omega)$ and $M_2 = (\lambda f . f \ulcorner 0 \urcorner
\Omega)$, as in the previous example.  Evaluating in the context $[](\lambda x . \mathcal{A} x)$
gives:\\
\[
 M_1 (\lambda x . \mathcal{A} x) = (\lambda f . \Omega) (\lambda x . \mathcal{A} x) \rhd
\Omega [f \mapsto (\lambda x . \mathcal{A} x)]
\]
\\
which diverges, and:\\
\[
 M_2 (\lambda x . \mathcal{A} x) = (\lambda f . f \ulcorner 0 \urcorner \Omega) (\lambda x .
\mathcal{A} x) \rhd (\lambda x . \mathcal{A} x) \ulcorner 0 \urcorner \Omega \rhd
(\mathcal{A} \ulcorner 0 \urcorner) \Omega \rhd \ulcorner 0 \urcorner
\]
\\
demonstrating that $M_1 \not{\equiv_v} M_2$.$\blacksquare$\\
\\
This example thus shows that $\equiv_v \nsubseteq \equiv_a$, since observational equivalence
in PCF$_v$ does not imply observational equivalence in PCF+$\mathcal{A}$.

\subsection{The Call-with-Current-Continuation Operator}

Control can be further extended using the $\mathcal{K}$-expression, which applies its argument
to an abstraction of the surrounding context (i.e. continuation).  The behavior of $\mathcal{C}$ is
summarized in the following evaluation rule:\\
\[
 E[\mathcal{K} M] \rhd E[M \lambda x . \mathcal{A} E[x]]
\]
\\
\textbf{Example:} Let $M_1$ and $M_2$ be defined as above, and let $C[] = \mathcal{K} (\lambda a . []a)$;
evaluation of $M_1$ gives:\\
\[
 \mathcal{K}(\lambda a . M_1\ a) = \mathcal{K} (\lambda a . (\lambda f . \Omega) a) \rhd
(\lambda f . \Omega)\ (\lambda x . \mathcal{A} E[x]) \rhd \Omega
\]
as before.  Evaluation of $M_2$ gives:\\
\begin{eqnarray*}
 \mathcal{K}(\lambda a . M_2\ a) \rhd\\
(\lambda a . M_2\ a)\ (\lambda x . \mathcal{A} E[x]) \rhd\\
M_2\ (\lambda x . \mathcal{A} E[x]) =\\
(\lambda f . f\ \ulcorner 0 \urcorner \Omega) (\lambda x . \mathcal{A} E[x]) \rhd\\
(\lambda x . \mathcal{A} E[x]) \ulcorner 0 \urcorner \Omega \rhd\\
(\mathcal{A} E[\ulcorner 0 \urcorner]) \Omega \rhd\\
 \ulcorner 0 \urcorner\\
\end{eqnarray*}
$\blacksquare$\\
\\

Note that this example is very similar to the preceding one, and that the operational distinction
between terms comes from the behavior of the abort operator, from which the call/cc operator is
derived.  The essential behavior to observe is that $\mathcal{C}$ immediately transfers control
to the evaluation of its argument, and subsequently returns control to the enclosing context with
the value of $eval(M)$.\\
\\
Let $eval_k$ and $\equiv_k$ respectively denote the evaluation function and the operational equivalence
relation for PCF$_v+\mathcal{K}$.  The preceding example exhibits a context for which two terms
that are members of $equiv_v$ are not members of $equiv_k$, whence $\equiv_v \nsubseteq \equiv_k$.\\
\\
Let PCF$_v-C$ denote the language obtained by extending PCF with both $\mathcal{A}$ and $\mathcal{C}$.

\end{document}
