\documentclass{beamer}

\usepackage[utf8x]{inputenc}
\usepackage{default}

\usepackage{fancyvrb}

\newcommand\codeHighlightRed[1]{\textcolor[rgb]{1,0,0}{\textbf{#1}}}
\newcommand\codeHighlightBlue[1]{\textcolor[rgb]{0,0,1}{\textbf{#1}}}
\newcommand\codeHighlightGreen[1]{\textcolor[rgb]{0.1,0.7,0.1}{\textbf{#1}}}
\newcommand\codeHighlightViolet[1]{\textcolor[rgb]{0.6, 0, 0.6}{\textbf{#1}}}
\newcommand\codeHighlightOrange[1]{\textcolor[rgb]{0.9, 0.3, 0}{\textbf{#1}}}

\newcommand\vbind[0]{\codeHighlightGreen{>>=}}
\newcommand\mret[0]{\codeHighlightGreen{return}}

\newcommand\hlam[0]{\ensuremath{\backslash}}
\newcommand\harr[0]{\codeHighlightOrange{->}}

\title{Using the Reactive Monad to Specify Network Protocols: Research Questions}

\author{Benjamin Schulz}

\begin{document}

\maketitle

\begin{frame}{Dear sirs,}

\begin{structure}{Some observations:}
\begin{itemize}

\item{Principals in a network communication have separate, local behaviors that interact in a
structured way that must satisfy some global behavior described by a network protocol.}

 \item{The thread-handler relation used for kernels naturally generalizes to a method for relating
local behaviors to global behaviors in a distributed system.}

\item{There is some work on high-level, verifiable protocol specification languages
(e.g. HLPSL by [Chevalier et al, 2004]), but not too much.}
\end{itemize}
\end{structure}

\emph{\color{red}{Does specifying the roles of a protocol as different reactive monads offer some advantage
in implementing or verifying protocols?}}


\end{frame}


\begin{frame}{Simple example: Narrative Spec}
 
\begin{structure}{Client}

\begin{itemize}
 \item {Makes a time-stamped query;}
 \item{Waits for a reply;}
 \item{If fault or out-of-sequence reply, tries again.}
\end{itemize}

 
\end{structure}

\begin{structure}{Server}

\begin{itemize}
\item{Listens for a query;}

\item{If a query is heard and has the right stamp, send back a reply with a matching stamp;}

\item{Otherwise, reply with 'fault'.}
\end{itemize}

\end{structure}


\end{frame}


\begin{frame}[fragile]{Simple example: Monadic Spec}

\begin{structure}{Client Implementation:}
\begin{small}
\begin{Verbatim}[commandchars=\\\{\}]
client = query \codeHighlightGreen{>>=} \hlam p \harr
         \codeHighlightRed{step_Re} (get) \codeHighlightGreen{>>=} \hlam n \harr
         \codeHighlightBlue{if} (reply n p)
         \codeHighlightBlue{then} (\mret p)
         \codeHighlightBlue{else} (dountil (\hlam p \harr reply n p) retry)

\end{Verbatim}
\end{small}
\end{structure}

\begin{structure}{Server Implementation:}
\begin{small}
\begin{Verbatim}[commandchars=\\\{\}]
server = listen \codeHighlightGreen{>>=} \hlam p \harr
         \codeHighlightRed{step_Re} (get \vbind \hlam s \harr put (s + 1) >> \mret s) \hlam s \harr
         (
            \codeHighlightBlue{if} (stampof p) == (s + 1)
            \codeHighlightBlue{then} send (Reply s)
            \codeHighlightBlue{else} send Fault

         ) \codeHighlightGreen{>>} server
\end{Verbatim}
\end{small}
\end{structure}

\end{frame}

\begin{frame}[fragile]{Monads for Roles, Types for Role Relationships}

\begin{structure}{States of the Channel}
 \begin{Verbatim}[commandchars=\\\{\}]
  \codeHighlightViolet{data} Event a = Msg a | Working | Listen

  step_Re x =
    P (Working, \hlam _ \harr x \vbind \hlam v \harr \mret (\mret v))
 \end{Verbatim}

\end{structure}


\begin{structure}{Messages in Client-Server}

 \begin{Verbatim}[commandchars=\\\{\}]
  \codeHighlightViolet{data} Req = Query Int | Retry Int
  \codeHighlightViolet{data} Rsp = Reply Int | Fault
 \end{Verbatim}

\end{structure}


\begin{structure}{Roles in Client-Server}

 \begin{Verbatim}[commandchars=\\\{\}]
  \codeHighlightViolet{type} Role q p m a = ReactT (Event q) (Event p) m a
  \codeHighlightViolet{type} Client a = Role Req Rsp (StateT Int Id) a
  \codeHighlightViolet{type} Server a = Role Rsp Req (StateT Int Id) a
 \end{Verbatim}

\end{structure}

\end{frame}

\begin{frame}[fragile]{Using handlers to put together a global view}
 
\begin{small}
\begin{Verbatim}[commandchars=\\\{\}]
 toplevel c s =
   \codeHighlightBlue{case} c \codeHighlightBlue{of}
     D _      \harr \mret ()
     P (q, _) \harr \codeHighlightBlue{case} s \codeHighlightBlue{of}
                   P (p, _)  \harr handle_c p c \vbind \hlam c' ->
                                handle_s q s \vbind \hlam s' ->
                                toplevel c' s'

                   _         \harr toplevel c s
\end{Verbatim}
\end{small}

\begin{structure}{What this specification says}
 \begin{itemize}
  \item {The client always acts first;}
  \item {The session is over when the client is done;}
  \item {Messages are assumed to be passed without external interference;}
 \end{itemize}

\end{structure}


\end{frame}


\begin{frame}[fragile]{Using handlers to express more varied behaviors}

 \begin{small}
\begin{Verbatim}[commandchars=\\\{\}]
 toplevel c s =
   \codeHighlightBlue{case} c \codeHighlightBlue{of}
     D _      \harr \mret ()
     P (q, _) \harr \codeHighlightBlue{case} s \codeHighlightBlue{of}
                   P (p, _)  \harr \codeHighlightRed{attack p} \vbind \hlam p' \harr
                                handle_c \codeHighlightRed{p'} c \vbind \hlam c' \harr
                                handle_s q s \vbind \hlam s' \harr
                                toplevel c' s'

                   _         \harr toplevel c s
\end{Verbatim}
\end{small}

\begin{structure}{Modeling vulernability in the global behavior}
 \begin{itemize}
  \item {Handler now states that the server's response \codeHighlightRed{p} can be modified in transit;}
  \item {Modified message is passed to the client;}
  \item {The handler now characterizes a set of traces with a specific vulnerability;}
 \end{itemize}

\end{structure}


\end{frame}

\begin{frame}{Summary: Future Questions}
 \begin{itemize}
  \item {Suppose all actions of type \texttt{\codeHighlightViolet{Role}} can be compiled into libraries implementing
   the protocol; does their relationship to the hanlder offer a verification advantage?}

  \item{The trace model exemplified above allows for stuttering; can this feature be used to
  formally express timing, fairness and liveness conditions like susceptibility to DoS attacks?}
 \end{itemize}

\end{frame}




\end{document}
