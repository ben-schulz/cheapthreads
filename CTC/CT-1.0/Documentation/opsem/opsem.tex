
\documentclass{article}

\usepackage{amsmath}
\usepackage{proof}
\usepackage{semantic}

\newcommand{\sem}[1]{|[ #1 |]}
\newcommand{\trans}[1]{\Longrightarrow_{#1}}
\newcommand{\state}[1]{\langle #1 \rangle}
\newcommand{\compile}[2]{\lceil #2 \rceil_{#1}}
\newcommand{\atom}[1]{\texttt{atom\{ #1 \}}}
\newcommand{\throw}[1]{\texttt{throw\{ #1 \}}}
\newcommand{\nullbind}[0]{\gg}
\newcommand{\bottom}[0]{\perp}

\newcommand{\forget}[1]{}

\title{Oh snap, son!  It's an OPSEM for RSI!}


\begin{document}

\maketitle

\section{Definition of the Abstract Machine}

	The RSI abstract machine has five types of states:
	
\begin{itemize}

\item{Top-level control: \[ \state{\kappa, \state{\rho, \theta, \iota, \sigma, \epsilon}}_M \]}

\item{Stream sequencing: \[ \state{C, \state{\rho, \theta, \iota, \sigma, \epsilon}}_R\]}

\item{Stream action:} \[ \state{A, \rho, \theta, \state{\kappa, \iota, \sigma, \epsilon}}_A \]

\item{Imperative sequencing: \[  \state{S, \rho, \state{\kappa, \theta, \iota, \sigma, \epsilon}}_K \]}

\item{Imperative action: \[ \state{U, \rho, S, \state{\kappa, \theta, \iota, \sigma, \epsilon}}_V \]}


\end{itemize}

Where:

\begin{itemize}
\item{$\kappa$ denotes a location label;}
\item{$\rho$ is the environment of variable bindings;}
\item{$\theta$ is the asynchronous request mask bit;}
\item{$\iota$ is the interrupt mask bit;}
\item{$\sigma$ is the loop-exit stack;}
\item{$\epsilon$ is the caller stack;}
\end{itemize}

\section{Transition Rules}

\subsection{Initial}
%%%% initial
\[
  \infer[(init)]{\state{\mu(C), \state{\rho_{init}, 1, 1, \emptyset, \emptyset}}_M}{\state{C, \mu}_{init}}
\]
\\

\subsection{Top-Level Control}
%%%% fetch-execute rule
\[
  \infer[(fetch-exec)]{\state{\mu(\kappa), \state{\rho, \theta, \iota, \sigma, \epsilon}}_R}{\state{\kappa, \state{\rho, \theta, \iota, \sigma, \epsilon}}_M}
\]
\\

\subsection{Atomic Actions}
%%%% atom rule:

\[
  \infer[(atom)]{\state{S, \rho, \state{\kappa, \theta, \iota, \sigma, \epsilon}}_K}{\state{\texttt{atom \{}S\texttt{\}}, \rho, \theta, \state{\kappa, \iota, \sigma, \epsilon}}_A}
\]
\\

%%%% throw rule:
\[
  \infer[(throw)]{\state{S, \rho, \state{\kappa, 0, 1, \sigma, \epsilon}}_K}{\state{\texttt{throw \{}S\texttt{\}}, \rho, 1, \state{\kappa, \iota, \sigma, \epsilon}}_A}
\]
\\

\[
  \infer[(throw-mask)]{\state{\kappa, \state{\rho, 1, \iota, \sigma, \epsilon}}_M}{\state{\texttt{throw \{}S\texttt{\}}, \rho, 0, \state{\kappa, \iota, \sigma, \epsilon}}_A}
\]

\subsection{Stream Execution}
%%%% sequence rule:
\[
  \infer[(seq)]{\state{C, \rho, \theta, \state{\mu(T), \iota, \sigma, \epsilon}}_A}{\state{C\texttt{;}T, \state{\rho, \theta, \iota, \sigma, \epsilon}}_R}
\]

%%%%% branch-on-zero:
\[
  \infer[(branch:\ zero)]{\state{C\prime\texttt{; }T, \state{\rho, \theta, \iota, \sigma, \epsilon}}_R}{\state{\texttt{bz } 0\ C\ C\prime\ T, \state{\rho, \theta, \iota, \sigma, \epsilon}}_R}
\]
\\
\[
  \infer[(branch:\ non-zero)]{\state{C\texttt{; }T, \state{\rho, \theta, \iota, \sigma, \epsilon}}_R}{\state{\texttt{bz } 1\ C\ C\prime\ T, \state{\rho, \theta, \iota, \sigma, \epsilon}}_R}
\]
\\

%%%% loop
\[
  \infer[(loop\ entry)]{\state{C, \state{\rho, \theta, \iota, (\mu(C) :: \sigma), (\mu(C\prime) :: \epsilon)}}_R}{\state{\texttt{loop }C\ C\prime, \state{\rho, \theta, \iota, \sigma, \epsilon}}_R}
\]
\\

\[
  \infer[(loop\ iteration)]{\state{\kappa, \state{\rho, \theta, \iota, (\kappa :: \sigma), \epsilon}}_M}{\state{\texttt{done}, \state{\rho, \theta, \iota, (\kappa :: \sigma), \epsilon}}_R}
\]

%%%% break
\[
  \infer[(loop\ exit)]{\state{\kappa, \state{\rho[\texttt{ret} \mapsto eval(\rho, v)], \theta, \iota, \sigma, \epsilon}}_M}{\state{\texttt{break }v, \state{\rho, \theta, \iota, (\kappa :: \sigma), \epsilon}}_R}
\]
\\

%%%% switch-R
\[
  \infer[(non-reactive\ switch)]{\state{\rho(X), \state{\rho, \theta, 1, \sigma, (\mu(T) :: \epsilon)}}_M}{\state{\texttt{swr } X\ T, \state{\rho, \theta, \iota, \sigma, \epsilon}}_R}
\]

%%%% switch-T
\[
  \infer[(reactive\ switch)]{\state{\rho(X), \state{\rho, \theta, 0, \sigma, (\mu(T) :: \epsilon)}}_M}{\state{\texttt{swt } X\ T, \state{\rho, \theta, \iota, \sigma, \epsilon}}_R}
\]

%%%% response-catch
\[
  \infer[(response\ catch)]{\state{C, \state{\rho[X \mapsto \rho(\texttt{rsp})], \theta, \iota, \sigma, \epsilon}}_R}{\state{\texttt{catch } X\ C, \state{\rho, \theta, \iota, \sigma, \epsilon}}_R}
\]


%%%% non-local jump
\[
  \infer[(jump)]{\state{\rho(X), \state{\rho, \theta, \iota, \sigma, \epsilon}}_M}{\state{\texttt{jmp } X, \state{\rho, \theta, \iota, \sigma, \epsilon}}_R}
\]
\\

%%%% return to caller
\[
  \infer[(call-back)]{\state{\kappa, \state{\rho, \theta, \iota, \sigma, \epsilon}}_M}{\state{\texttt{halt}, \state{\rho, \theta, \iota, \sigma, (\kappa :: \epsilon)}}_R}
\]

%%%% system exit
\[
  \infer[(system\ exit)]{\state{}_{final}}{\state{\texttt{halt}, \state{\rho, \theta, \iota, \emptyset, \emptyset}}_R}
\]

\subsection{Imperative Sequencing}

\[
  \infer[(imperative\ sequencing)]{\state{A, \rho, S, \state{\kappa, \theta, \iota, \sigma, \epsilon}}_V}{\state{A\texttt{;}S, \rho, \state{\kappa, \theta, \iota, \sigma, \epsilon}}_K}
\]
\\

\[
  \infer[(local\ branch: zero)]{\state{S\prime, \rho, \state{\kappa, \theta, \iota, \sigma, \epsilon}}_K}{\state{\texttt{bz }0\ S\ S\prime, \rho, \state{\kappa, \theta, \iota, \sigma, \epsilon}}_K}
\]
\\

\[
  \infer[(local\ branch: non-zero)]{\state{S, \rho, \state{\kappa, \theta, \iota, \sigma, \epsilon}}_K}{\state{\texttt{bz }1\ S\ S\prime, \rho, \state{\kappa, \theta, \iota, \sigma, \epsilon}}_K}
\]
\\

\[
  \infer[(action\ terminate)]{\state{\kappa, \state{\rho, \theta, \iota, \sigma, \epsilon}}_M}{\state{\texttt{end}, \rho, \state{\kappa, \theta, \iota, \sigma, \epsilon}}_K}
\]

\subsection{Imperative Actions}

%%%% assignment

\[
  \infer[(assignment)]{\state{S, \rho[X \mapsto eval(\rho, v)], \state{\kappa, \theta, \iota, \sigma, \epsilon}}_K}{\state{X \texttt{:= } v, \rho, S, \state{\kappa, \theta, \iota, \sigma, \epsilon}}_V}
\]
\\


%%%% clear

\[
  \infer[(variable\ clear)]{\state{S, \rho[X \mapsto \bottom], \state{\kappa, \theta, \iota, \sigma, \epsilon}}_K}{\state{\texttt{clr } X, \rho, S, \state{\kappa, \theta, \iota, \sigma, \epsilon}}_V}
\]
\\

%%%% nop

\[
  \infer[(nop)]{\state{S, \rho, \state{\kappa, \theta, \iota, \sigma, \epsilon}}_K}{\state{\texttt{nop}, \rho, S, \state{\kappa, \theta, \iota, \sigma, \epsilon}}_V}
\]




% all of the remainder is old news:
%
% (2010.10.05) Schulz
%

\forget{

\section{Control Flow}
\begin{tabular}[t]{llll}



$\sem{ L\texttt{: atom\{}k\} }$ &=& $\state{L, \rho, 0, \sigma, \epsilon} \trans{K} \sem{k} \trans{R} \state{L, \rho\prime, 1, \sigma, \epsilon}$\\

$\sem{ L\texttt{: throw\{}k\} }$ &=& $\state{L, \rho, 0, (L\prime :: \sigma), \epsilon} \trans{K} \sem{k} \trans{R} \state{L\prime, \rho, 0, \sigma, \epsilon}$\\
\\
$\sem{A\texttt{; }C}$ &=& $\sem{\texttt{A}} \trans{R} \sem{\texttt{C}}$\\
\\
$\sem{L\texttt{: bz}\ x\ T\ F\texttt{; }C}$ &=&
$
\begin{cases}

\state{L, \rho, \iota, \sigma, \epsilon} \trans{R} \sem{F} \trans{R} \sem{C}\ ,\ eval(\rho, x) = 0\\
\state{L, \rho, \iota, \sigma, \epsilon} \trans{R} \sem{T} \trans{R} \sem{C}\ ,\ otherwise\\

\end{cases}
$ \\
\\
$\sem{L\texttt{: loop }R\texttt{; }C}$ &=& $\state{L, \rho, \iota, \sigma, (\mu(C) :: \epsilon)} \trans{R} \sem{R} \trans{R} \state{\mu(R), \rho\prime, \iota\prime, \sigma\prime, \epsilon\prime}$\\

$\sem{L\texttt{: break}\ x}$ &=& $\state{L, \rho, \iota, \sigma, (L\prime :: \epsilon)} \trans{R} \state{L\prime, \rho[ret \mapsto eval(\rho, x)], \iota, \sigma, \epsilon}$\\

$\sem{L\texttt{: swr }x\texttt{; }C}$ &=& $\state{L, \rho, \iota, \sigma, \epsilon} \trans{R} \state{\mu(eval(\rho, x)), \rho, \iota, (\mu(C) :: \sigma), \epsilon}$\\

$\sem{L\texttt{: swt }x\texttt{; }C}$ &=& $\state{L, \rho, \iota, \sigma, \epsilon} \trans{R} \state{\mu(eval(\rho, x)), \rho, 1, (\mu(C) :: \sigma), \epsilon}$\\

$\sem{L\texttt{: jmp }x}$ &=& $\state{L, \rho, \iota, \sigma, \epsilon} \trans{R} \state{\mu(eval(\rho, x)), \iota, \sigma, \epsilon}$\\

$\sem{L\texttt{: halt}}$ &=& $\state{L, \rho, \iota, \sigma, \epsilon} \trans{final} done$\\

\end{tabular}

\section{State Actions}

\begin{tabular}[t]{llll}

$\sem{S\texttt{; }S\prime}$ &=& $\sem{S} \trans{K} \sem{S\prime}$\\

$\sem{\texttt{bz}\ x\ T\ F\texttt{; }S}$ &=&
$
\begin{cases}

\state{\kappa, \rho, \iota, \sigma, \epsilon} \trans{K} \sem{F} \trans{K} \sem{S}\ ,\ eval(\rho, x) = 0\\
\state{\kappa, \rho, \iota, \sigma, \epsilon} \trans{K} \sem{T} \trans{K} \sem{S}\ ,\ otherwise\\

\end{cases}
$ \\
\\
$\sem{g \texttt{ := }x}$ &=& $\state{\kappa, \rho, \iota, \sigma, \epsilon} \trans{K} \state{\kappa, \rho[g \mapsto eval(\rho, x)], \iota, \sigma, \epsilon}$\\
\\
$\sem{\texttt{clr } g}$ &=& $\state{\kappa, \rho, \iota, \sigma, \epsilon}, \trans{K} \state{\kappa, \rho[g \mapsto \bottom], \iota, \sigma, \epsilon}$\\
\\
$\sem{\texttt{nop}}$ &=& $\state{\kappa, \rho, \iota, \sigma, \epsilon} \trans{K} \state{\kappa, \rho, \iota, \sigma, \epsilon}$\\

\end{tabular}

} % end forget


\end{document}