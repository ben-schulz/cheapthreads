\documentclass{article}

\usepackage{amsmath}
\usepackage{proof}

\title{}
\author{Benjamin Schulz}

\begin{document}

\begin{abstract}

To answer your question, yes we still amaze, rock, shock and put the whole
block in a daze.  Quit bein sarcastic, you're bein a mess.  Quit poppin that
junk -- we got your address.

\end{abstract}

\section{Source Language}

\subsection{Types}
\begin{tabular}[t]{lll}

 $\kappa$ &::=& $(r, ir)$\\
 $t$ &::=& \texttt{unit} $\vert$ \texttt{bool} $\vert$ \texttt{nat} $\vert$
           $(s + s)$ $\vert$
           $(s \times s)$ $\vert$ $(s \rightarrow_{ir} s)$
           $\vert$ \texttt{ref} $s$\\
 $s$ &::=& ($t$, $\kappa$)\\
\end{tabular}

Here, $r$ and $ir$ are members of a lattice of security levels, indicating
the \emph{reader} and \emph{indirect reader} privileges.  All terms have
both a \emph{value} type and a \emph{security type}, as reflected in the
mutual recursion between $t$ and $s$.  Functions may be side-effecting; the
permission level of these side-effects is indicated by the subscript. Mutable
reference types are indicated by \texttt{ref}, with the argument indicating
type of the referent.

\subsection{Values}


\begin{tabular}[t]{lll}
 $V\prime$ &::=& \texttt{()} $\vert$ (\texttt{inj}$_i$ $V$) $\vert$
                 $\langle V , V \rangle$
                 $\vert$ $(\lambda (x : s) . E)$
                 $l^s$\\
 $V$ &::=& $V\prime_\kappa$\\
\end{tabular}

Functions are first-class, and thus \emph{values} are mutually recursive with
\emph{terms}, defined below.  Reference cells are denoted by $l$, and must
be given an explicit type and explicit permissions, as indicated by the
superscript $s$.

\subsection{Terms}

\begin{tabular}[t]{lll}

 $E$ &::=& $x$ $\vert$ $V$ $\vert$ (\texttt{inj}$_i$ $E$)$_\kappa$ $\vert$
           $\langle E , E  \rangle_\kappa$ $\vert$ ($E$ $E$) $\vert$
           (\texttt{proj}$_i$ $E$)$_r$ $\vert$\\
           &&(\texttt{protect}$_{ir}$ $E$) $\vert$
           ($\mu (f : s) . E$)) $\vert$ \\
           &&(\texttt{case} $E$ \texttt{of}
            \texttt{inj}$_1$(x) . $E$ $\vert$ \texttt{inj}$_2$(x) . $E$)$_r$\\
           &&(\texttt{ref$_s$ $E$})$_{\kappa}$ $\vert$ ($E$ := $E$)$_r$ $\vert$
           (!$E$)$_r$ $\vert$\\

\end{tabular}

\section{Intermediate Language 1: CT-HASK}

\subsection{Language of Threads: Reactivity}

\begin{tabular}[t]{lll}
 
$T$ &::=& $A\ \star\ \lambda x . T$ $\vert$
          $A \gg$  $T$
          $\vert$ \texttt{if} $X$ \texttt{then} $M$ \texttt{else} $M$\\

$A$ &::=& $\eta\ X\ \vert $ \texttt{signal} $Q$\\
\\

$Q$ &::=& $\texttt{apply}\ X\ X$ $\vert$ \texttt{mkcl} $x\ T\ \vert$
          \texttt{mkreccl} $x\ T\ \vert$ \texttt{lkp} $x$\\
\\
$X$ &::=& $C\ \vert\ x\ \vert $ \texttt{inc} $X\ \vert $
          \texttt{dec} $X\ \vert X$ \texttt{==} $X$\\
          &&$\vert $ \texttt{isnum}\ $X \vert $ \texttt{isbool}\ $X \vert$
          \texttt{iscl}\ $X \vert $ \texttt{isreccl}\ $X$\\
          &&$\vert$ \textbf{cl} $<x, \rho, T> \vert $
          \textbf{reccl} $<x, \rho, T>$\\

$C$ &::=& 0 $\vert$ 1 $\vert$ ... $\vert$ \textbf{false} $\vert$ \textbf{true} 

\end{tabular}

\end{document}
